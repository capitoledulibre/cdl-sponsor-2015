\documentclass{cdl_sponsor}


%%%%%%%%%%%%%%%%%%%%%%%%%%%%%%%%%%%%%%%%
% Définition des variables de la classe cdl_sponsor

\DefTitre{Capitole du Libre 2014}
\DefSousTitre{Dossier de Sponsoring}
\DefAuteur{Toulibre}
\DefWeb{http://2014.capitoledulibre.org}

%%%%%%%%%%%%%%%%%%%%%%%%%%%%%%%%%%%%%%%%
% Informations du document (pour pdflatex)
\ifpdf
  \hypersetup{pdftitle={\SousTitre}}
  \hypersetup{pdfauthor={\Auteur}}
  \hypersetup{pdfsubject={\Titre}}
  \hypersetup{pdfcreator={\Auteur}}
  \hypersetup{pdfproducer={\Auteur}}
  \hypersetup{pdfkeywords={}}
\fi

\begin{document}
\CreerTitre{Images/titre2.jpg}{Benh \bsc{Lieu Song} -- CC-BY-SA 3.0}

\begin{Introduction}

L'association \textbf{Toulibre} organise le \textcolor{Cdl}{15 et 16 novembre 2014} la quatrième édition du \textbf{\g{Capitole du Libre}}, un événement consacré aux Logiciels Libres, orienté à la fois vers le grand public et le public spécialisé.

\Separateur

Des cycles de conférences grand public, techniques et multimédia ont lieu le samedi. Le dimanche est consacré à des ateliers pratiques.

\Separateur

Une \textbf{\textit{Install Party}} permet à tous de découvrir et d'installer un système Libre sur son ordinateur. Des stands de démo et d'animations sont proposés au public toute la journée du samedi. Un \textbf{\g{village du libre}} permet aux associations autour du libre de présenter leur activité.

\Separateur

Le \textbf{Capitole du Libre} est également l'occasion de réunir des communautés du Libre pour des conférences, \textit{lightning talks}, \textit{coding sprints}\dots ~ Le \textbf{Capitole du Libre} a accueilli plusieurs conférences depuis 2011 telles que \textbf{DrupalCamp}, \textbf{DjangoCon},  \textbf{FranceJS}, \textbf{LuaWorkshop}, \textbf{OpenStack} et \textbf{Akademy-FR}.

\Separateur

Vous trouverez plus d'informations sur le site internet : \url{\Web}.
\end{Introduction}

\section{Conférences, démonstrations et ateliers du Capitole du Libre}

Pendant tout un weekend, le Capitole du Libre propose conférences, animations et ateliers autour du Logiciel Libre et du bien commun, et accueille des événements de la communauté du Libre.

\subsection{Conférences et ateliers}

Les conférences et les ateliers qui seront programmés pourront aborder les thèmes suivants :
\begin{itemize}
\item[\logo] un cycle de \textbf{conférences grand public} couvrant des sujets comme les enjeux des Logiciels Libres, le Libre au-delà du Logiciel, Wikipédia, les aspects économiques ou sociaux du Logiciel Libre\dots ~ ;
\item[\logo] un thème \textbf{bureautique et multimédia}, couvrant des sujets tels que la retouche d'image, la modélisation 3D, la musique assistée par ordinateur, la bureautique\dots ~ ;
\item[\logo] un thème \textbf{technique} couvrant des sujets de développement logiciels, d'embarqué, d'administration système ou réseau\dots ~ ;
\item[\logo] un thème \textbf{Internet Libre} couvrant les solutions qui permettent de maîtriser ses données personnelles sur la Toile ;
\item[\logo] un thème \textbf{Arduino} et \textbf{Open Hardware} sur les aspects matériel et montages électroniques libres ;
\item[\logo] un thème \textbf{DevOps} sur les problématiques d'automatisation du déploiement d'applications.
\end{itemize}

\Separateur

Dans chaque thème, les conférences proposées seront de 20 minutes à une heure, et permettent de parler de plus de sujets dans une durée adaptée.

\Separateur

Depuis l'édition de 2011, les conférences ont lieu le samedi, et le dimanche est consacré aux ateliers pratiques, aussi bien pour le grand public que pour un public averti. Les thèmes pourront évoluer en fonction des propositions reçues.

\Separateur

Les années précédentes sont notamment intervenus Nicolas \bsc{Barcet} de \textbf{eNovance \& RedHat}, Benjamin \bsc{Bayart} de \textbf{FDN}, Stéphane \bsc{Bortzmeyer} de l'\textbf{AFNIC}, Adrienne \bsc{Charmet Alix} de \textbf{Wikimedia France}, Alix \bsc{Cazenave} et Frédéric \bsc{Couchet} de l'\textbf{April}, Claire \bsc{Gallon} de \textbf{LiberTIC}, Alexis \bsc{Kauffmann} et Pierre-Yves \bsc{Gosset} de \textbf{Framasoft}, Sandrine \bsc{Mathon} de \textbf{Toulouse Métropole}, Lucas \bsc{Nussbaum} et Stefano \bsc{Zacchiroli} de \textbf{Debian}, François \bsc{Pelligrini}, Paul \bsc{Rouget} de la \textbf{Mozilla Fondation}, Christophe \bsc{Sauthier} d'\textbf{Ubuntu-fr}, Jérémie \bsc{Zimmermann} de \textbf{La Quadrature du Net}\dots

\subsection{Espace de stands et d'échanges}

Lieu de passage du public, le grand hall de l'\bsc{Enseeiht} est un espace dédié aux stands pour les organisations, associations ou entreprises.

\Separateur

Il est possible également de poser du matériel de communication tel que \g{kakemono}, présentoirs\dots

\Separateur

En plus des stands, un espace convivial sera aménagé afin de permettre rencontres et discussions.

\newpage
\subsection{Install Party, Lan Party et espace démonstrations}

L'\textit{install party} est un événement de promotion et de démocratisation des Logiciels Libres auprès du grand public, et un événement de rencontre des acteurs de la communauté du Logiciel Libre.

\Separateur

Organisé chaque année depuis 2008, il est intégré dans le Capitole du Libre depuis 2011.

\Separateur
 
Cet événement propose :
\begin{itemize}
\item[\logo] un espace de démonstration de Logiciels Libres, où les visiteurs pourront poser leurs questions ou participer à des mini-ateliers ;
\item[\logo] Des mini-présentations des différentes distributions Libres proposées (Ubuntu, Fedora, OpenSuse, LinuxMint\dots) ;
\item[\logo] une \textit{install party}, permettant au grand public de trouver de l'aide pour installer des logiciels et distributions Libres sur leur propre ordinateur.
\end{itemize}

\subsection{Nouveautés de cette édition}

Grande nouveauté du Capitole du Libre 2014 : un \textcolor{Cdl}{village associatif} va être organisé permettant ainsi de présenter de nombreuses distributions \bsc{Gnu}/Linux. Plusieurs groupes d'utilisateurs du libre seront aussi au rendez-vous pour présenter au public les dernières nouveautés technologiques libres du moment.

\section{Événements hébergés}

\subsection{Akademy-fr}

\ImageDroitebis{Images/klogo-official-lineart_simple-128x128.png}
%
Akademy-fr est la déclinaison française de la conférence KDE annuelle. KDE est une communauté internationale produisant un ensemble d'applications multi-plateformes et notamment un environnement de bureau nommé Plasma Desktop.

\Separateur

Le but de l'Akademy-fr est d'améliorer la promotion de KDE au niveau de la France, à l'aide de conférences en français et d'ateliers orientés contribution.

\subsection{Hackfest LibreOffice}

\ImageDroite{Images/LO.png}

LibreOffice est une suite bureautique libre et gratuite ; son interface claire et ses puissants outils vous permettent de libérer votre créativité et de développer votre productivité.

\Separateur

LibreOffice intègre plusieurs applications qui en font la plus puissante suite bureautique Libre et Open Source du marché. Le développement est ouvert à de nouveaux talents et de nouvelles idées, et le logiciel est testé et utilisé quotidiennement par une importante communauté d'utilisateurs dévoués.

\Separateur

Dans un \textit{Hackfest} les contributeurs LibreOffice se réunissent afin de coordonner dans un temps donné et dans une atmosphère détendue, le développement du produit et faire avancer le projet. Au sein de LibreOffice, cela passe par une communauté internationale de développeurs, designers, traducteurs et \g{QA triagers}.

\Separateur 

C'est une occasion idéale pour commencer ou continuer à participer à l'un des plus grands projets Open Source dans l'action !

\section{Conditions de sponsoring}

\textbf{Toulibre} offre la possibilité à des entreprises d'associer leur nom à l'événement \g{Capitole du Libre} qui touchera à la fois le grand public et le public spécialisé en informatique. Auprès du grand public, l'image des sponsors sera associée à un événement s'intéressant aux enjeux éthiques et sociaux du numérique. Auprès du public spécialisé, les sponsors se feront connaître comme acteurs du monde du Logiciel Libre.\par 
Vous pouvez participer à l'événement financièrement ou bien en prenant en charge un poste de dépense (comme l'impression des affiches, le repas du dimanche\dots).

\Separateur

Selon que l'entreprise qui sponsorise est un grand compte ou une PME, le niveau des montants est différent (que se soit en service ou en monétaire).

\Separateur

En échange du sponsoring du Capitole du Libre par les entreprises associées à l'événement, Toulibre s'engage à :

\begin{center}
\arrayrulecolor{Cdl} 
\begin{tabular}{|p{10cm}ccc|}
\hline
\rowcolor{Cdl} & \textbf{Bronze} & \textbf{Argent} & \textbf{Or} \\
\hline\hline
{\hfill\textit{Grand compte}} & \SI{300}{\euro} & \SI{700}{\euro} & \SI{1500}{\euro} \\ 

{\hfill\textit{PME}} & \SI{200}{\euro} & \SI{500}{\euro} & \SI{900}{\euro} \\ 
\hline\hline
Logo sur l'affiche de l'événement\textcolor{Cdl}{*} & \textcolor{Cdl}{\ding{'064}} & \textcolor{Cdl}{\ding{'064}} & \textcolor{Cdl}{\ding{'064}} \\ 

Logo sur le site internet dédié à l'événement\textcolor{Cdl}{*} & \textcolor{Cdl}{\ding{'064}} & \textcolor{Cdl}{\ding{'064}} & \textcolor{Cdl}{\ding{'064}} \\ 

Logo au début de chaque vidéo\textcolor{Cdl}{*} & \textcolor{Cdl}{\ding{'064}} & \textcolor{Cdl}{\ding{'064}} & \textcolor{Cdl}{\ding{'064}} \\ 

Badges sponsors & \textcolor{Cdl}{\ding{'064}} & \textcolor{Cdl}{\ding{'064}} & \textcolor{Cdl}{\ding{'064}} \\ 

Remerciements lors de la conférence de clôture de l'événement & \textcolor{Cdl}{\ding{'064}} & \textcolor{Cdl}{\ding{'064}} & \textcolor{Cdl}{\ding{'064}} \\ 

Logo sur le programme papier distribué aux participants\textcolor{Cdl}{*} &  & \textcolor{Cdl}{\ding{'064}} & \textcolor{Cdl}{\ding{'064}} \\ 

Logo sur les flyers de l'événement\textcolor{Cdl}{*} &  & \textcolor{Cdl}{\ding{'064}} & \textcolor{Cdl}{\ding{'064}} \\ 


Texte de description sur le programme papier distribué\textcolor{Cdl}{**} &  & \textcolor{Cdl}{\textbf{120}} & \textcolor{Cdl}{\textbf{380}} \\

Espace pour un stand &  &  & \textcolor{Cdl}{\ding{'064}} \\ 

Interview sur le site du Capitole du Libre &  &  & \textcolor{Cdl}{\ding{'064}} \\
\hline
\multicolumn{4}{r}{\textcolor{Cdl}{*} \textit{logo de taille différente : or plus grand que argent plus grand que bronze}}\\
\multicolumn{4}{r}{\textcolor{Cdl}{**} \textit{en nombre de caractères}}\\

\end{tabular}
\end{center}

\subsection{Les dépenses}

Comme les années passées, les conférenciers pour les présentations et les ateliers seront défrayés. Votre participation permet également de financer le chauffage des locaux de l'\bsc{Enseeiht} que nous occupons tout le weekend, ainsi qu'un poste de secours obligatoire pour un événement de cette ampleur. L'apéritif dînatoire du samedi soir est en partie financé par la participation libre des convives. Tout le public y est invité.

\Separateur

Le Capitole du Libre est organisé exclusivement par les \textbf{bénévoles} de l'association \textbf{Toulibre} et des \textbf{clubs info, vidéo et animation de l'\bsc{ENSEEIHT}}.

\Separateur
Le budget du Capitole du Libre s'élève à environ \SI{10000}{\euro}.

\newpage
\subsection{Ils nous ont soutenu en 2013}

\begin{center}
{\Large \textcolor{Cdl}{Partenaires Or}}

\urllogoau{http://www.enovance.com}{enovance.png}\hspace{1cm}
\urllogoau{www.kdab.com}{kdab.png}\hspace{1cm}
\urllogoau{http://www.logilab.fr}{logilab.png} \\
\urllogoau{http://makina-corpus.com}{makina-corpus.png}\hspace{1cm}
\urllogoau{http://boards.openpandora.org/page/homepage.html}{openpandora.png}\hspace{1cm}
\urllogoau{http://www.sierrawireless.com}{sierra-wireless.png}

\Separateur

{\Large \textcolor{Cdl}{Partenaires Argent}}

\Separateur

\urllogoag{http://www.objectif-libre.com}{objectif-libre.png}

\Separateur

{\Large \textcolor{Cdl}{Partenaires Bronze}}

\Separateur

\urllogobr{http://blue-mind.net}{bluemind.jpg} \hspace{1cm}
\urllogobr{http://free-electrons.com}{free-electrons.png}\hspace{1cm}
\urllogobr{http://www.nfrance.com}{nfrance-conseil.png} \\
\urllogobr{http://osones.com}{osones.png} \hspace{1cm}
\urllogobr{http://www.solulibre.com}{solulibre.png}
\end{center}


\section{Contact}

Pour toutes questions relatives au sponsoring du Capitole du Libre :
\begin{itemize}
\item[\logo] écrivez à \href{mailto:contact@capitoledulibre.org}{\nolinkurl{contact@capitoledulibre.org}}
\end{itemize}
\paragraph{Contact presse :}
\begin{itemize}
\item[\logo] écrivez à \href{mailto:comm@capitoledulibre.org}{\nolinkurl{comm@capitoledulibre.org}}
\end{itemize}

\paragraph{Photos première page  :} \href{https://commons.wikimedia.org/wiki/File\%3AToulouse_Capitole_Night_Wikimedia_Commons.jpg}{Benh \bsc{Lieu Song} -- CC-BY-SA 3.0}



\end{document}
