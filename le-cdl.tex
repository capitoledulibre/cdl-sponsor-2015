% Partie le Capitole du Libre

Le Capitole du Libre est un événement autour du Logiciel Libre à 
Toulouse, qui a lieu tous les ans en novembre à l'ENSEEIHT.

Chaque année ce sont plusieurs conférences sur des sujets variés 
aussi bien techniques que grand public qui se déroulent en même temps 
sur tout un weekend. Des ateliers pratiques sont également proposés 
à des publics divers.

\Separateur

Le \textbf{Capitole du Libre} est également l'occasion de réunir des 
communautés du Libre pour des conférences, \textit{lightning talks}, 
\textit{coding sprints}\dots ~ Le \textbf{Capitole du Libre} a 
accueilli plusieurs conférences depuis 2011 telles que 
\textbf{DrupalCamp}, \textbf{DjangoCon},  \textbf{FranceJS}, 
\textbf{LuaWorkshop}, \textbf{OpenStack}, \textbf{Hackfest LibreOffice} 
et \textbf{Akademy-FR}.

\Separateur

Un village associatif permet aux associations du libre de présenter 
leurs projets, telles que OpenStreetMap, Wikimedia France, 
Tetaneutral.net, ou Liberté0.

\subsection{Le Capitole du Libre en chiffres}

En 2014, pas moins de \textbf{63 conférences} et \textbf{21 ateliers} 
ont été proposés durant tout le weekend. Les stands associatifs ne 
sont pas en reste puisque \textbf{22 associations} ou projets libres 
étaient représentés.

\Separateur

L'événement est basé uniquement sur le bénévolat, durant les mois 
de préparation et pendant tout le week-end, pendant lequel plus de 60 
bénévoles sont présents pour accueillir le public, préparer le 
programme, filmer les conférences, ou encore aider à installer des 
logiciels libres.

\subsection{Un événement accessible}

\begin{minipage}{0.38\textwidth}
\begin{center}
\photo{clavier-braille-photo-jeanZ.jpg}{0.8\textwidth}
\end{center}
\end{minipage}
\begin{minipage}{0.62\textwidth}
2014 a été l'occasion de proposer aux visiteurs des programmes 
imprimés en braille, ce qui a permis à plusieurs déficients visuels 
de profiter pleinement de l'événement. Dans la continuité, en 2015 
nous souhaitons proposer la traduction de certains conférences en LSF.
\Separateur
L'accessibilité de notre événement est un aspect que nous souhaitons 
développer afin d'inclure tous les publics, et nous veillons à ce que 
tous les ateliers et conférences soient accessibles pour les personnes 
à mobilité réduite.
\end{minipage}

\newpage

\subsection{Des intervenants de qualité}

Le Capitole du Libre a accueilli des personnalités du Logiciel Libre telles que Adrienne \bsc{Charmet} ou Jérémie \bsc{Zimmermann} de \textbf{La Quadrature du Net}, Benjamin \bsc{Bayart} de \textbf{FDN}, Stéphane \bsc{Bortzmeyer} de l'\textbf{AFNIC}, Alix \bsc{Cazenave} et Frédéric \bsc{Couchet} de l'\textbf{April}, Claire \bsc{Gallon} de \textbf{LiberTIC}, Alexis \bsc{Kauffmann} et Pierre-Yves \bsc{Gosset} de \textbf{Framasoft}, Sandrine \bsc{Mathon} de \textbf{Toulouse Métropole}, Nicolas \bsc{Barcet} de \textbf{eNovance \& RedHat}, Lucas \bsc{Nussbaum} et Stefano \bsc{Zacchiroli} de \textbf{Debian}, François \bsc{Pelligrini}, Paul \bsc{Rouget} de la \textbf{Mozilla Fondation}, Christophe \bsc{Sauthier} d'\textbf{Ubuntu-fr}, etc.

\subsection{Des ateliers et animations pour expérimenter}

%\InsertImage{CDL2014-2953-imprimante3D-photo-jeanZ.JPG}{0.33334}{r}

\begin{minipage}{0.6\textwidth}
Et comme rien ne vaut l'expérimentation, tout au long du weekend sont 
proposées au public des animations et ateliers ludiques. Cela est 
l'occasion de découvrir les imprimantes 3D, comment héberger soi-même son 
site web, contribuer à OpenStreetMap, ou encore programmer son jeu vidéo.
\end{minipage}
\begin{minipage}{0.4\textwidth}
\photo{CDL2014-2953-imprimante3D-photo-jeanZ.JPG}{0.8\textwidth}
\end{minipage}

