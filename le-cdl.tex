% Partie le Capitole du Libre

Le \href{http://capitoledulibre.org/}{Capitole du Libre} est un
 événement tous publics autour du logiciel libre,
 organisé à Toulouse chaque année depuis 2009 au mois de novembre
 par l'association \href{http://toulibre.org/}{Toulibre}.

Le \textbf{Capitole du Libre} est l'un des événements autour du logiciel libre
 qui fait référence en dehors de Paris. Le public et les orateurs y viennent plus nombreux chaque année.

Chaque année ce sont
 \href{http://2014.capitoledulibre.org/programme/conferences/list/}{plusieurs conférences}
 sur des sujets variés aussi bien techniques que grand public qui se
 déroulent en parallèle tout au long du week-end.
Des \href{http://2014.capitoledulibre.org/programme/ateliers/list/}{ateliers pratiques}
 variés sont également proposés au public.

\Separateur

Le \textbf{Capitole du Libre} est également l'occasion de réunir des 
communautés du Libre pour des conférences, \textit{lightning talks}, 
\textit{coding sprints}\dots ~ Le \textbf{Capitole du Libre} a 
accueilli plusieurs événements depuis 2011 tels que :
\begin{itemize}[label=$\bullet$]
\item \textbf{DrupalCamp} en 2011
\item \textbf{DjangoCon} en 2012
\item \textbf{FranceJS} en 2013
\item \textbf{LuaWorkshop} en 2013
\item \textbf{OpenStack} en 2013
\item \textbf{Hackfest LibreOffice}  en 2014
\item \textbf{Akademy-FR} depuis la première édition !
\end{itemize}

\Separateur

Un village associatif permet également aux associations du libre de
 présenter leurs projets, telles que \href{http://liberte0.org/wiki/index.php?title=Accueil}{Liberté0},
 \href{http://openstreetmap.fr/}{OpenStreetMap France},
 \href{http://wikimedia.fr/}{Wikimédia France},
 \href{http://framasoft.net/}{Framasoft},
 ou encore \href{http://tetaneutral.net/}{Tetaneutral.net}.

\begin{center}
\image{cdl-amphi.jpg}{1\textwidth}
\textit{Amphithéâtre comble pour Stéphane Bortzmayer de l'AFNIC}
\end{center}

\subsection{Le Capitole du Libre en chiffres}

\begin{minipage}{0.6\textwidth}
En 2014, Capitole du Libre c'était :
\begin{itemize}[label=$\bullet$]
\item \textbf{plus de 1000 visiteurs} ;
\item \textbf{63 conférences} ;
\item \textbf{50 heures de conférences} filmées et disponibles en ligne ;
\item \textbf{9 flux de conférence en parallèle} tout le week-end ;
\item \textbf{21 ateliers} ;
\item \textbf{22 associations} représentées dans le village associatif ;
\item \textbf{\SI{2}{\tera o} de vidéos} sous licence libre une fois traitées.
\end{itemize}
\end{minipage}
\begin{minipage}{0.4\textwidth}
\begin{center}
\image{hall-1-photo-n7.jpg}{1\textwidth}
\end{center}
\end{minipage}

\Separateur

L'événement est basé uniquement sur le bénévolat, durant les mois 
de préparation et pendant tout le week-end, pendant lequel plus de \textbf{60 
bénévoles} des associations et clubs techniques de l'ENSEEIHT
sont présents pour accueillir le public, préparer le 
programme, filmer les conférences, ou encore aider à installer des 
logiciels libres.

\subsection{L'édition 2015}

Comme tous les ans, nous couvrirons des thématiques à la fois techniques et grand 
public:

\begin{itemize}[label=$\bullet$]
\item \textbf{DevOps} : inaugurée en 2014, sera renouvelée et abordera Docker, le \textit{cloud} libre ;
\item \textbf{Internet Of Things \& Do It Yourself} : (re)découverte des Arduino, Raspberry Pi, objets connectés ;
\item l’\textbf{Internet libre} : autohébergement, sécurité et vie privée sur l’Internet ;
\item \textbf{Technologies Web} : NodeJS, ReactJS, AngularJS, le Web change tous les jours ;
\item \textbf{Animation 2D \& 3D} : Blender, Krita, Gimp, Inkscape, Synfig, la réalisation d'œuvres artistiques avec des outils libres ;

\item \textbf{AkademyFR} : le rendez-vous de la communauté francophone de KDE et de Qt ;
\item et nos grands classiques : culture libre et multimédia, etc.
\end{itemize}

\subsection{Un événement accessible}

\begin{minipage}{0.4\textwidth}
\begin{center}
\image{clavier-braille-photo-jeanZ.jpg}{0.8\textwidth}
\end{center}
\end{minipage}
\begin{minipage}{0.6\textwidth}
L'accessibilité de notre événement est un aspect que nous souhaitons
 développer afin d'inclure tous les publics, et nous veillons à ce que
 tous les ateliers et conférences soient accessibles pour les personnes
 à mobilité réduite.

2014 a été l'occasion de proposer aux visiteurs des programmes
 imprimés en braille, ce qui a permis à plusieurs déficients visuels
 de profiter pleinement de l'événement. Dans la continuité, en 2015
 nous souhaitons proposer la traduction de certaines conférences en
 langue des signes.
\end{minipage}

\subsection{Des intervenants de qualité}

Chaque année, de nombreuses personnalités participent au Capitole du Libre. Parmi elles, on peut compter :

\begin{itemize}[label=$\bullet$]
\item Benjamin \bsc{Bayart} de \textbf{FDN}, Adrienne \bsc{Charmet} et Jérémie \bsc{Zimermann} de \textbf{La Quadrature du Net} qui ont assuré à plusieurs reprises les conférences de clôture ;
\item Stéphane \bsc{Bortzmeyer} de l'\textbf{AFNIC} qui a présenté les dessous de l'Internet mondial ;
\item et Sandrine \bsc{Mathon} de \textbf{Toulouse Métropole} qui présente chaque année les avancées de notre métropole en terme d'\textit{open data}.
\end{itemize} 

\subsection{Des ateliers et animations pour expérimenter}

\begin{minipage}{0.6\textwidth}
Comme rien ne vaut l'expérimentation, tout au long du week-end sont 
proposées au public des animations et ateliers ludiques. Cela est 
l'occasion de découvrir les imprimantes 3D, comment héberger soi-même son 
site Web, contribuer à OpenStreetMap, ou encore programmer son propre jeu vidéo.
\end{minipage}
\begin{minipage}{0.4\textwidth}
\begin{center}
\image{CDL2014-2953-imprimante3D-photo-jeanZ.JPG}{0.7\textwidth}
\end{center}
\end{minipage}

