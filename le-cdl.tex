% Partie le Capitole du Libre

Le \href{http://capitoledulibre.org}{Capitole du Libre} est un
 évènement tout public de promotion des logiciels libres,
 organisé par l’association \href{http://toulibre.org/}{Toulibre}.
 Il se déroule à Toulouse, chaque année depuis 2009, sur un 
 weekend du mois de novembre.

Grâce à une programmation variée et de qualité, 
 le \textbf{Capitole du Libre} est aujourd’hui une référence
 parmi les manifestations consacrées aux logiciels libres. Le
 public et les orateurs y viennent plus nombreux — et
 de plusieurs pays voisins — chaque année.

Tout au long du weekend, des
 \href{http://2014.capitoledulibre.org/programme/conferences/list/}{conférences}
 et des \href{http://2014.capitoledulibre.org/programme/ateliers/list/}{ateliers pratiques}
 sur des sujets variés, aussi bien techniques que grand public, se déroulent en parallèle.


%%% \textit{freetux} : Grosses redites dans ces trois premiers paragraphes. À reformuler.

 
\Separateur

Le \textbf{Capitole du Libre} est également l'occasion de réunir des 
communautés du libre pour des conférences, \textit{lightning talks}, 
\textit{coding sprints}\dots ~ Le \textbf{Capitole du Libre} a 
accueilli plusieurs évènements depuis 2011 tels que :
\begin{itemize}[label=$\bullet$]
\item \textbf{DrupalCamp} en 2011 ;
\item \textbf{DjangoCon} en 2012 ;
\item \textbf{FranceJS} en 2013 ;
\item \textbf{LuaWorkshop} en 2013 ;
\item \textbf{OpenStack} en 2013 ;
\item \textbf{Hackfest LibreOffice}  en 2014 ;
\item \textbf{Akademy-FR} depuis la première édition !
\end{itemize}

\Separateur

Un village associatif permet également de présenter les projets des associations du libre : \href{http://liberte0.org/wiki/index.php?title=Accueil}{Liberté0},
 \href{http://wikimedia.fr/}{Wikimédia France},
 \href{http://openstreetmap.fr/}{OpenStreetMap France},
 \href{http://framasoft.net/}{Framasoft}
 ou encore \href{http://tetaneutral.net/}{Tetaneutral.net}.

\begin{center}
\image{cdl-amphi-cc-by-guillaume-paumier.jpg}{1\textwidth}
\textit{Amphithéâtre comble pour Stéphane \bsc{Bortzmayer} de l'AFNIC}
\end{center}

\subsection{Quelques chiffres}

\begin{minipage}{0.6\textwidth}
En 2014, le Capitole du Libre c'était :
\begin{itemize}[label=$\bullet$]
\item \textbf{plus de \num{1000} visiteurs} ;
\item \textbf{63~conférences} ;
\item \textbf{50~heures de conférences} filmées et disponibles en ligne ;
\item \textbf{9~flux de conférences en parallèle} tout le weekend ;
\item \textbf{21~ateliers} ;
\item \textbf{22~associations} représentées dans le village associatif ;
\item \textbf{\SI{2}{\tera o} de vidéos} sous licence libre une fois traitées.
\end{itemize}
\end{minipage}
\begin{minipage}{0.4\textwidth}
\begin{center}
\image{hall-1-photo-n7.jpg}{1\textwidth}
\end{center}
\end{minipage}

\Separateur

L’évènement est basé uniquement sur le bénévolat.
 Durant les mois de préparation et pendant tout le weekend,
 plus de \textbf{60~bénévoles} des associations et des clubs techniques
 de l’INP-ENSEEIHT se sont mobilisés pour l’élaboration du programme,
 l’accueil du public, la captation des conférences ou encore l’aide
 à l’installation des logiciels libres.


\subsection{L'édition 2015}

Comme tous les ans, nous couvrirons des thématiques à la fois techniques et grand 
public :

\begin{itemize}[label=$\bullet$]
\item \textbf{DevOps} : inaugurée en 2014, sera renouvelée et abordera Docker, le \textit{cloud} libre ;
\item \textbf{Internet Of Things \& Do It Yourself} : (re)découverte des Arduino, Raspberry Pi, objets connectés ;
\item l’\textbf{Internet libre} : autohébergement, sécurité et vie privée sur l’Internet ;
\item \textbf{Technologies Web} : NodeJS, ReactJS, AngularJS, le Web change tous les jours ;
\item \textbf{Animation 2D \& 3D} : Blender, Krita, Gimp, Inkscape, Synfig, la réalisation d'œuvres artistiques avec des outils libres ;

\item \textbf{AkademyFR} : le rendez-vous de la communauté francophone de KDE et de Qt ;
\item et nos grands classiques : culture libre et multimédia, etc.
\end{itemize}

\subsection{Un évènement accessible}

\begin{minipage}{0.4\textwidth}
\begin{center}
\image{clavier-braille-photo-jeanZ.jpg}{0.8\textwidth}
\end{center}
\end{minipage}
\begin{minipage}{0.6\textwidth}
L'accessibilité de notre évènement est un aspect que nous souhaitons
 développer afin d'inclure tous les publics, et nous veillons à ce que
 tous les ateliers et conférences soient accessibles pour les personnes
 à mobilité réduite.

2014 a été l'occasion de proposer aux visiteurs des programmes
 imprimés en braille, ce qui a permis à plusieurs déficients visuels
 de profiter pleinement de l'évènement. Dans la continuité, en 2015
 nous souhaitons proposer la traduction de certaines conférences en
 langue des signes.
\end{minipage}

\subsection{Des intervenants de qualité}

Chaque année, de nombreuses personnalités participent au Capitole du Libre. Parmi elles, on peut compter :

\begin{itemize}[label=$\bullet$]
\item Benjamin \bsc{Bayart} de \textbf{FDN}, Adrienne \bsc{Charmet} et Jérémie \bsc{Zimmermann} de \textbf{La Quadrature du Net} qui ont assuré à plusieurs reprises les conférences de clôture ;
\item Stéphane \bsc{Bortzmeyer} de l'\textbf{AFNIC} qui a présenté les dessous de l'Internet mondial ;
\item Sandrine \bsc{Mathon} de \textbf{Toulouse Métropole} qui présente chaque année les avancées de notre métropole en termes d'\textit{open data}\dots
\end{itemize} 

\subsection{Des ateliers et animations pour expérimenter}

\begin{minipage}{0.6\textwidth}

Comme rien ne vaut l'expérimentation, des animations et ateliers ludiques sont proposés tout au long du weekend. Le public est invité à venir découvrir l’impression 3D, l’autohébergement de site Web, la contribution à OpenStreetMap ou encore la programmation de son propre jeu vidéo.

\end{minipage}
\begin{minipage}{0.4\textwidth}
\begin{center}
\image{CDL2014-2953-imprimante3D-photo-jeanZ.JPG}{0.7\textwidth}
\end{center}
\end{minipage}

