% Partie Capitole du Libre résumé

L'association Toulibre organise les \textbf{samedi 21 et dimanche 22 novembre 2015} la cinquième
édition du \href{http://capitoledulibre.org}{Capitole du Libre}, un évènement consacré aux Logiciels Libres et à la culture libre, orienté à la fois vers le grand public et le public spécialisé.

Cet événement est l'occasion de découvrir comment \textbf{libérer son ordinateur} et \textbf{reprendre la main sur sa vie privée numérique}, au travers de conférences, d'ateliers et de démonstration.
Chacun pourra discuter avec des acteurs du logiciel libre dans le \textbf{village associatif}.

Les précédentes éditions ont attiré jusqu'à 1000 personnes !

Des cycles de conférences thématiques auront lieu durant tout le weekend,
avec des intervenants tels que Benjamin Bayart de la {Fédération FDN des FAI associatifs},
\textbf{Adrienne Charmet} de {La Quadrature du Net},
\textbf{Amaelle Guiton} de {Libération},
\textbf{Tristan Nitot} de {CozyCloud},
\textbf{Stéphane Bortzmeyer} de l'{Afnic},
\textbf{Laurent Chemla}, co-fondateur de {Gandi},
et \textbf{Lionel Maurel} de {SavoirsComm1}, juriste et documentaliste.

Les thèmes des conférences et des ateliers permettront de découvrir ou d'approfondir les divers aspects du libre, tels que les enjeux de société, les aspects juridiques, la culture libre ou la protection de la vie privée numérique, mais aussi l'informatique embarquée et IoT, la création graphique ou la CAO, ou encore les technologies du web ou la bureautique.

\textbf{L'entrée est libre et gratuite}. Des programmes en braille seront disponibles à l'accueil, et certaines conférences pourront être interprétées en LSF.

Dates : samedi 21 et dimanche 22 novembre 2015
Lieu : \textbf{ENSEEIHT} Institut National Polytechnique de Toulouse
26, rue Riquet, 31 000 Toulouse
Site Web : \href{http://capitoledulibre.org}{http://2015.capitoledulibre.org}
