% Partie Capitole du Libre résumé

\Separateur

\par{\fontsize{12pt}{18pt}\selectfont
  L'association Toulibre organise les \textbf{samedi 21 et dimanche 22 novembre 2015} la cinquième
  édition du \href{http://capitoledulibre.org}{Capitole du Libre},
  un évènement consacré aux Logiciels Libres et à la culture libre,
  orienté à la fois vers le grand public et le public spécialisé.
}

\Separateur

Cet événement est l'occasion de découvrir comment \textbf{libérer son ordinateur}
et \textbf{reprendre la main sur sa vie privée numérique},
au travers de conférences, d'ateliers et de démonstration.
Chacun pourra discuter avec des acteurs du logiciel libre dans le \textbf{village associatif}.

Les précédentes éditions ont attiré jusqu'à 1000 personnes !

\Separateur

Des cycles de conférences thématiques auront lieu durant tout le weekend,
avec des intervenants tels que \textbf{\mbox{Benjamin Bayart}} de la \href{http://ffdn.org/}{Fédération FDN des FAI associatifs},
\textbf{\mbox{Adrienne Charmet}} de \href{http://laquadrature.net/}{La Quadrature du Net},
\textbf{\mbox{Amaelle Guiton}} de \href{http://www.liberation.fr/auteur/15260-amaelle-guiton}{Libération},
\textbf{\mbox{Tristan Nitot}} de \href{https://cozy.io/fr/}{CozyCloud},
\textbf{\href{http://www.bortzmeyer.org/}{\mbox{Stéphane Bortzmeyer}}} de l'\href{}{Afnic},
\textbf{\mbox{Laurent Chemla}}, co-fondateur de \href{http://gandi.net/}{Gandi},
et \textbf{\href{http://scinfolex.com/}{\mbox{Lionel Maurel}}} de \href{http://savoirscom1.info/}{SavoirsComm1}, juriste et documentaliste.

\Separateur

Les thèmes des conférences et des ateliers permettront de découvrir ou d'approfondir
les divers aspects du libre, tels que les \textbf{enjeux de société}, les \textbf{aspects juridiques},
la \textbf{culture libre} ou la \textbf{protection de la vie privée numérique},
mais aussi l'\textbf{informatique embarquée} et \textbf{IoT}, la \textbf{création graphique} ou la \textbf{CAO},
ou encore \textbf{les technologies du web} ou la \textbf{bureautique}.

\Separateur

\textbf{L'entrée est libre et gratuite}.

\Separateur

Des programmes en braille seront disponibles à l'accueil, et certaines conférences pourront être interprétées en LSF.

\begin{itemize}[label=$\bullet$]
\item Dates : samedi 21 et dimanche 22 novembre 2015
\item Lieu : \textbf{ENSEEIHT} Institut National Polytechnique de Toulouse
26, rue Riquet, 31 000 Toulouse
\item Site Web : \href{http://capitoledulibre.org}{http://2015.capitoledulibre.org}
\end{itemize}
