% Partie Capitole du Libre résumé

\Separateur
\par{\fontsize{14pt}{22pt}\selectfont
  L'association Toulibre organise les \textbf{samedi 19 et dimanche 20 novembre 2016} la cinquième
  édition du \href{http://capitoledulibre.org}{Capitole du Libre},
  un évènement consacré aux Logiciels Libres et à la culture libre,
  orienté à la fois vers le grand public et le public spécialisé.
  L'événement attire chaque année un peu plus de 1000 personnes.
}

\begin{itemize}[label=$\bullet$]
\item Dates : samedi 19 et dimanche 20 novembre 2016
\item Lieu : \textbf{ENSEEIHT} Institut National Polytechnique de Toulouse
26, rue Riquet, 31 000 Toulouse
\item Site Web : \href{http://capitoledulibre.org}{http://2016.capitoledulibre.org}
\end{itemize}

\Separateur

Cet événement est l'occasion de découvrir comment \textbf{libérer son ordinateur}
et \textbf{reprendre la main sur sa vie privée numérique},
au travers de conférences, d'ateliers et de démonstration.
Chacun pourra discuter avec des acteurs du logiciel libre dans le \textbf{village associatif}.

Les précédentes éditions ont attiré jusqu'à 1000 personnes !

\Separateur

Des cycles de conférences thématiques auront lieu durant tout le weekend,
avec des intervenants tels que \textbf{\mbox{Benjamin Bayart}} de la \href{https://www.ffdn.org/}{Fédération FDN des FAI associatifs},
\textbf{\href{https://scinfolex.com/}{\mbox{Lionel Maurel}}} de \href{https://www.savoirscom1.info/}{SavoirsCom1}, juriste et documentaliste,
des personnes de \textbf{\href{https://www.mozilla.org/}{Mozilla}} et de \textbf{\href{https://cozy.io/}{Cozy Cloud}}.

\Separateur

Les thèmes des conférences et des ateliers permettront de découvrir ou d'approfondir
les divers aspects du logiciel libre et du libre en général, tels que 
l'\textbf{informatique embarquée} et l'\textbf{IoT} (\g{Internet des Objets}), la \textbf{création graphique},
\textbf{les technologies du web} ou l'\textbf{utilisation quotidienne des logiciels libre}, la tendance \textbf{DevOps},
mais aussi les \textbf{enjeux de société}, les \textbf{aspects juridiques},
la \textbf{culture libre} ou la \textbf{protection de la vie privée numérique}.

\Separateur

\textbf{L'entrée est libre et gratuite}.

