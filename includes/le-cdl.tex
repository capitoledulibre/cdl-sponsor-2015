% Partie le Capitole du Libre

Le \href{http://capitoledulibre.org}{Capitole du Libre} est un
 évènement tout public de promotion des logiciels libres,
 organisé par l’association \href{http://toulibre.org/}{Toulibre}.
 Il se déroule à Toulouse, chaque année depuis 2009, sur un 
 weekend du mois de novembre.

Grâce à une programmation variée et de qualité, 
 le \textbf{Capitole du Libre} est aujourd’hui une référence
 parmi les manifestations consacrées aux logiciels libres. Le
 public et les orateurs y viennent plus nombreux — et
 de plusieurs pays voisins — chaque année.

Tout au long du weekend, des
 \href{https://2015.capitoledulibre.org/programme/conferences/}{conférences}
 et des \href{https://2015.capitoledulibre.org/programme/ateliers/}{ateliers pratiques}
 sur des sujets variés, aussi bien techniques que grand public, se déroulent en parallèle.

 
\Separateur

Le \textbf{Capitole du Libre} est également l'occasion de réunir des 
communautés du libre pour des conférences, \textit{lightning talks}, 
\textit{coding sprints}\dots ~ Le \textbf{Capitole du Libre} a 
accueilli plusieurs évènements depuis 2011 tels que :
\begin{itemize}[label=$\bullet$]
\item \textbf{DrupalCamp} en 2011 ;
\item \textbf{DjangoCon} en 2012 ;
\item \textbf{FranceJS} en 2013 ;
\item \textbf{LuaWorkshop} en 2013 ;
\item \textbf{OpenStack} en 2013 ;
\item \textbf{Hackfest LibreOffice}  en 2014 ;
\item \textbf{Akademy-FR} depuis la première édition !
\end{itemize}

\Separateur

Un village associatif permet également de présenter les projets des associations du libre : \href{http://liberte0.org/wiki/index.php?title=Accueil}{Liberté0},
 \href{http://wikimedia.fr/}{Wikimédia France},
 \href{https://openstreetmap.fr/}{OpenStreetMap France},
 \href{https://framasoft.org/}{Framasoft}
 ou encore \href{https://tetaneutral.net/}{Tetaneutral.net}.

\begin{center}
\image{cdl-amphi-cc-by-guillaume-paumier.jpg}{1\textwidth}
\textit{Amphithéâtre comble pour Stéphane \bsc{Bortzmayer} de l'AFNIC}
\end{center}

\Separateur

L’évènement est basé uniquement sur le bénévolat.
 Durant les mois de préparation et pendant tout le weekend,
 \textbf{100~bénévoles} des associations et des clubs techniques
 de l’INP-ENSEEIHT se sont mobilisés pour l’élaboration du programme,
 l’accueil du public, la captation des conférences ou encore l’aide
 à l’installation des logiciels libres.


\subsection{L'édition 2016}

Pas moins de 14 thématiques aussi bien techniques que grand public seront abordées cette année:

\begin{itemize}[label=$\bullet$]
\item \textbf{Découverte du libre} : Pour découvrir l'écosystème du logiciel libre;
\item \textbf{Enjeux du libre} : Qu'est-ce que le libre transforme dans notre société ;
\item \textbf{Libertés et vie privée} : Autohébergement, sécurité et vie privée sur l’Internet ;
\item \textbf{Culture libre} : Les arts et la culture libres, la MAO, la vidéo ;
\item \textbf{Logiciels libres au quotidien} : Environnement de travail, bureautique, utilitaires ;
\item \textbf{Technique} : Développement, bases de données, pour la beauté du code ;
\item \textbf{DevOps} : le monde de l'infrastructure est en plein bouleversement: Cloud, Docker, automatisation, etc ;
\item \textbf{Objets connectés \& embarqués} : (re)découverte des Arduino, Raspberry Pi, objets connectés ;
\item \textbf{Technologies Web} : NodeJS, ReactJS, AngularJS, le Web change tous les jours ;
\item \textbf{3D, Création graphique et multimedia} : Blender, Krita, Gimp, Inkscape, Synfig, la réalisation d’œuvres graphiques ou animées avec des outils libres ;
\item \textbf{C++ / Qt} : Tout ce qui touche au développement C++ / Qt et KDE ;
\item \textbf{Communautés du libre} : Le logiciel libre, ce n'est pas que la technique. Retrouvez toute une série de conférence autour de la vie des communautés du logiciel libre ;
\item \textbf{Communs} : Tout ce qui touche aux Communs: immatériel, matériel. Le libre au delà du logiciel
\end{itemize}

Cette année, le Capitole du Libre héberge également la \textbf{8e rencontre des monaies libres}, rencontre des développeurs des monaies libres.

Le programme complet est à découvrir sur \href{https://2016.capitoledulibre.org/programme.html}{notre site web}.

\subsection{Quelques chiffres sur l'édition 2016}

L'édition 2016, après un an d'absence est une édition record:

\begin{itemize}[label=$\bullet$]
\item \textbf{109} speakers dont plus de la moitié se déplace pour l'occasion
\item \textbf{99} conférences
\item \textbf{27} ateliers
\item \textbf{25} stands dans le village associatif
\item \textbf{2} keynotes
\item \textbf{2} food trucks
\item \textbf{100} bénévoles
\item \textbf{1} boutique de goodies !
\end{itemize}

\subsection{Deux keynotes}

Le samedi, deux keynotes seront assurés par de prestigieux orateurs et qui abordent autant les problèmes techniques que sociétals:

\begin{itemize}[label=$\bullet$]
\item \textbf{Un Web sous surveillance} par Thibault Jouannic: L'espionnage des internautes comme modèle économique dominant sur le Web : souriez, vous êtes trackés !
\item \textbf{Travailler ensemble pour la défense des libertés} par Benjamin Bayart, Pierre-Yves GOSSET et piks3l: Quels enjeux pour les libertés sur Internet ? Quel chemin avons-nous déjà accompli ? Plus qu'un bilan, la Fédération FDN, Framasoft et La Quadrature du Net vous parleront de travail collaboratif et du futur.
\end{itemize}

\subsection{Un évènement accessible}

\begin{minipage}{0.4\textwidth}
\begin{center}
\image{clavier-braille-photo-jeanZ.jpg}{0.8\textwidth}
\end{center}
\end{minipage}
\begin{minipage}{0.6\textwidth}
L'accessibilité de notre évènement est un aspect que nous souhaitons
 développer afin d'inclure tous les publics, et nous veillons à ce que
 tous les ateliers et conférences soient accessibles pour les personnes
 à mobilité réduite.

2014 a été l'occasion de proposer aux visiteurs des programmes
 imprimés en braille, ce qui a permis à plusieurs déficients visuels
 de profiter pleinement de l'évènement. Dans la continuité, en 2016
 nous souhaitons continuer de rendre notre conférence aussi accessible
 que possible.
\end{minipage}

\subsection{Des intervenants de qualité}

Chaque année, de nombreuses personnalités participent au Capitole du Libre. Parmi elles, nous pouvons compter :

\begin{itemize}[label=$\bullet$]
\item Benjamin \bsc{Bayart} de \textbf{FDN} qui a assuré à plusieurs reprises les conférences de clôture ;
\item Pierre-Yves \bsc{Gosset} de \textbf{Framasoft} ;
\item Christopher \bsc{Talib} de \textbf{la Quadrature du Net} ;
\item Julien \bsc{Simon} de \textbf{Amazon Web Services} ;
\item Adrien \bsc{Blind} de \textbf{Docker Paris} ;
\item Plusieurs intervenants de Mozilla ;
\end{itemize} 

\subsection{Des ateliers et animations pour expérimenter}

\begin{minipage}{0.6\textwidth}

Comme rien ne vaut l'expérimentation, des animations et ateliers ludiques sont proposés tout au long du weekend. Le public est invité à venir découvrir l'édition graphique 2D et 3D avec des logiciels libre, la programmation et les techniques autour de la programmation, le Cloud Open Source, les logiciels libre permettant de garantir votre vie privée.
Ces ateliers sont ouverts à tous et permettent d'apprendre et de gagner en expertise sur ces logiciels libres.

\end{minipage}
\begin{minipage}{0.4\textwidth}
\begin{center}
\image{CDL2014-2953-imprimante3D-photo-jeanZ.JPG}{0.7\textwidth}
\end{center}
\end{minipage}


\subsection{Précédentes éditions}

\subsubsection{L'édition 2015}

\label{ed2015}

Suite aux attentats de novembre 2015, aucun événement public n'a pu se tenir dans les locaux de l'INP-ENSEEIHT. Ainsi nous avons du annuler le Capitole du Libre tel que nous le connaissions. Mais ne rien faire n'était pas non plus satisfaisant. 

Par la suite nous avons donc réduit l'événement dans plusieurs lieux toulousains alternatifs :

\begin{itemize}[label=$\bullet$]
\item  \href{http://www.coworking-toulouse.com}{Étincelle} pour la thématique DevOps ;
\item  \href{http://www.artilect.fr}{Artilect} a hébergé les thématiques 3D et Création libre, coordonné par le BUG ;
\item À \href{https://www.ekito.fr}{Ekito} eu les lieux les conférences sur le Logiciel libre : développement, contribution ;
\item \href{http://mixart-myrys.org/le-lieu/}{Mix'art Myrys} a accueilli le plus gros de l'événement : Enjeux du libre et société.
\end{itemize}

Le Capitole du libre 2015 fut, pour l'occasion rebaptisé \href{http://bazardulibre.org}{Bazar du libre}.



\subsubsection{L'édition 2014}

\begin{minipage}{0.6\textwidth}
En 2014, le Capitole du Libre c'était :
\begin{itemize}[label=$\bullet$]
\item \textbf{plus de \num{1000} visiteurs} ;
\item \textbf{63~conférences} ;
\item \textbf{50~heures de conférences} filmées et disponibles en ligne ;
\item \textbf{9~flux de conférences en parallèle} tout le weekend ;
\item \textbf{21~ateliers} ;
\item \textbf{22~associations} représentées dans le village associatif ;
\item \textbf{\SI{2}{\tera o} de vidéos} sous licence libre une fois traitées.
\end{itemize}
\end{minipage}
\begin{minipage}{0.4\textwidth}
\begin{center}
\image{hall-1-photo-n7.jpg}{1\textwidth}
\end{center}
\end{minipage}


