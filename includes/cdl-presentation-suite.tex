% Partie Capitole du Libre résumé

\Separateur

Cet événement est l'occasion de découvrir comment \textbf{libérer son ordinateur}
et \textbf{reprendre la main sur sa vie privée numérique},
au travers de conférences, d'ateliers et de démonstrations.
Chacun pourra discuter avec des acteurs du logiciel libre dans le \textbf{village associatif}.

Les précédentes éditions ont attiré jusqu'à \num{1000}~personnes !

\Separateur

Des cycles de conférences thématiques auront lieu durant tout le weekend,
avec des intervenants tels que \textbf{\mbox{Benjamin Bayart}} de la \href{https://ffdn.org/}{Fédération FDN des FAI associatifs},
et \textbf{\href{https://scinfolex.com/}{Lionel Maurel}} de \href{https://savoirscom1.info/}{SavoirsCom1}, juriste et documentaliste,
des personnes de \textbf{\href{https://www.mozilla.org/}{Mozilla}} et de \textbf{\href{https://cozy.io/}{Cozy Cloud}}.

\Separateur

Les thèmes des conférences et des ateliers permettront de découvrir ou d'approfondir
les divers aspects du logiciel libre et du libre en général, tels que 
l'\textbf{informatique embarquée} et l'\textbf{IoT} (\g{Internet des Objets}), la \textbf{création graphique},
\textbf{les technologies du web} ou l'\textbf{utilisation quotidienne des logiciels libre}, la tendance \textbf{DevOps},
mais aussi les \textbf{enjeux de société}, les \textbf{aspects juridiques},
la \textbf{culture libre} ou la \textbf{protection de la vie privée numérique}.

\Separateur

\textbf{L'entrée est libre et gratuite}. Retrouvez le \href{https://2016.capitoledulibre.org/programme.html}{programme complet}.

