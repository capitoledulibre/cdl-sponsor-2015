% Partie Toulibre

\textbf{Toulibre} est une association d'utilisateurs et de développeurs de logiciels libres de la région toulousaine. Elle organise des actions visant à promouvoir, développer et démocratiser les logiciels libres dans la région Midi-Pyrénées. L'association est également concernée par la promotion des œuvres diffusées sous licences libres, et se place dans une perspective d'éducation populaire. Enfin, Toulibre se veut également être un support pour les communautés locales du libre.

\Separateur

Toulibre organise des évènements ouverts à tous :
\begin{itemize}[label=$\bullet$]
\item des rencontres régulières favorisant la découverte des logiciels libres (présentation, aide à l’utilisation et à l’installation) ;
\item des ateliers mensuels, permettant le développement et/ou la 
pratique autonome de certains logiciels ou technologies ;
\item des manifestations ponctuelles, dont la plus importante en termes 
de public et de portée est le \textbf{Capitole du Libre}.
\end{itemize}

L'association intervient également lors d'évènements ou dans des \textbf{espaces publics numériques} (EPN) pour proposer des ateliers sur des logiciels libres ou de l'aide pour les utiliser et les installer, par exemple :
\begin{itemize}[label=$\bullet$]
\item au \textbf{Forum numérique des seniors} en 2013 et 2014 ;
\item dans les médiathèques de Tournefeuille, Blagnac et Colomiers ;
\item au cinéma Utopia de Tournefeuille.
\end{itemize}
