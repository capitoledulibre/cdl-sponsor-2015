\documentclass{cdl_sponsor}


%%%%%%%%%%%%%%%%%%%%%%%%%%%%%%%%%%%%%%%%
% Définition des variables de la classe cdl_sponsor

\DefTitre{Capitole du Libre 2015}
\DefSousTitre{Dossier de Sponsoring}
\DefAuteur{Toulibre}
\DefWeb{http://2015.capitoledulibre.org}

%%%%%%%%%%%%%%%%%%%%%%%%%%%%%%%%%%%%%%%%
% Informations du document (pour pdflatex)
\ifpdf
  \hypersetup{pdftitle={\SousTitre}}
  \hypersetup{pdfauthor={\Auteur}}
  \hypersetup{pdfsubject={\Titre}}
  \hypersetup{pdfcreator={\Auteur}}
  \hypersetup{pdfproducer={\Auteur}}
  \hypersetup{pdfkeywords={}}
\fi

\begin{document}

\thispagestyle{empty} % Remove page numbering on this page

\ThisURCornerWallPaper{1}{Images/bgfirstpage.eps}

%----------------------------------------------------------------------------------------
%	TITLE SECTION
%----------------------------------------------------------------------------------------

\parbox[t]{1.0\textwidth}{
	\flushright \fontsize{45pt}{60pt}\selectfont % The first argument for fontsize is the font size of the text and the second is the line spacing - you may need to play with these for your particular title
	\vspace*{0.7cm} % Space between the start of the title and the top of the grey box
		
	\textcolor{white}{
		\hfill Capitole du Libre 2015 \\
		\fontsize{36pt}{50pt}\selectfont{
			\hfill Dossier Sponsoring \\
			\hfill Toulibre \\
		}
	}
	\par
	
	\vspace*{0.7cm} % Space between the end of the title and the bottom of the grey box
}

%----------------------------------------------------------------------------------------
%	Image
%----------------------------------------------------------------------------------------

\InsertPhoto{Images/titre2.jpg}{Benh \bsc{Lieu Song} -- CC-BY-SA 3.0}

%----------------------------------------------------------------------------------------

\vfill % Space between the title box and author information

%----------------------------------------------------------------------------------------
%	AUTHOR NAME AND INFORMATION SECTION
%----------------------------------------------------------------------------------------
{\centering \large 
\hfill Association Toulibre \\
\hfill \url{http://toulibre.org} \\
\hfill \url{http://capitoledulibre.org} \\
\hfill \url{contact@capitoledulibre.org} \\
%\HRule{1pt}} % Horizontal line, thickness changed here
}
%----------------------------------------------------------------------------------------
\clearpage

\CenterWallPaper{0.95}{Images/fond.eps}

\section{Présentation des logiciels libres}

	% Partie logiciels libres

\citation{Je peux expliquer le logiciel libre en trois mots : \\ Liberté, Égalité, Fraternité.}{Richard Matthew \bsc{Stallman}}


\section{Toulibre}

	% Partie Toulibre

\textbf{Toulibre} est une association d'utilisateurs(trices) et de développeurs(peuses) de Logiciels Libres de la région Toulousaine. Elle organise des actions visant à promouvoir, développer et démocratiser les Logiciels Libres dans la région Midi-Pyrénées. L'association est également concernée par la promotion des œuvres diffusées sous licence libre, et se place dans une perspective d'éducation populaire.
Toulibre se veut également être un support pour les communautés locales du libre.

\Separateur

Toulibre organise des évènements ouverts à tous :
\begin{itemize}[label=$\bullet$]
\item des rencontres régulières permettant la découverte des Logiciels Libres, l'aide à l'utilisation et l'installation de Logiciels Libres, la présentation d'un logiciel etc.
\item des ateliers mensuels, permettant le développement et/ou la pratique régulière de certains logiciels ou technologies
\item des manifestations ponctuelles, dont la plus importante en terme de public et de portée est le \textbf{Capitole du Libre}
\end{itemize}


\section{Le Capitole du Libre}

	% Partie le Capitole du Libre

Le Capitole du Libre est un événement tous publics autour du Logiciel Libre à Toulouse organisé depuis 2009 tous les ans par l'association Toulibre au mois de Novembre.

Le Capitole du Libre est un des événements autour du Logiciel Libre qui fait référence en dehors de Paris. Chaque année, le public et les orateurs viennent assister de plus en plus nombreux à la conférence. 

Chaque année ce sont plusieurs conférences sur des sujets variés 
aussi bien techniques que grand public qui se déroulent en même temps 
tout au long du week-end. Des ateliers pratiques sont également proposés 
à des publics divers.

\Separateur

Le \textbf{Capitole du Libre} est également l'occasion de réunir des 
communautés du Libre pour des conférences, \textit{lightning talks}, 
\textit{coding sprints}\dots ~ Le \textbf{Capitole du Libre} a 
accueilli plusieurs conférences depuis 2011 telles que:
\begin{itemize}[label=$\bullet$]
\item \textbf{DrupalCamp} en 2011
\item \textbf{DjangoCon} en 2011
\item \textbf{FranceJS} en 2012
\item \textbf{LuaWorkshop} en 2013
\item \textbf{OpenStack} en 2013
\item \textbf{Hackfest LibreOffice}  en 2014
\item \textbf{Akademy-FR} depuis la première édition!
\end{itemize}

\Separateur

Un village associatif permet aux associations du libre de présenter 
leurs projets, telles que OpenStreetMap, Wikimedia France, 
Tetaneutral.net, ou Liberté0.

\subsection{Le Capitole du Libre en chiffres}

En 2014, Capitole du Libre c'était:
\begin{itemize}[label=$\bullet$]
\item \textbf{plus de 1000 visiteurs}
\item \textbf{63 conférences}
\item \textbf{50 heures de conférences} filmées et disponibles sur le site web
\item \textbf{9 flux de conférence en parallèle} tout au long du week-end
\item \textbf{21 ateliers}
\item \textbf{22 associations} réprésentées dans le village associatif
\item \textbf{2 To de videos} sous licence libre une fois traitées
\end{itemize}

\Separateur

L'événement est basé uniquement sur le bénévolat, durant les mois 
de préparation et pendant tout le week-end, pendant lequel plus de 60 
bénévoles des différentes associations et des clubs techniques de l'ENSSEIHT
sont présents pour accueillir le public, préparer le 
programme, filmer les conférences, ou encore aider à installer des 
logiciels libres.

\subsection{Un événement accessible}

\begin{minipage}{0.38\textwidth}
\begin{center}
\photo{clavier-braille-photo-jeanZ.jpg}{0.8\textwidth}
\end{center}
\end{minipage}
\begin{minipage}{0.62\textwidth}
L'accessibilité de notre événement est un aspect que nous souhaitons 
développer afin d'inclure tous les publics, et nous veillons à ce que 
tous les ateliers et conférences soient accessibles pour les personnes 
à mobilité réduite.
\Separateur
2014 a été l'occasion de proposer aux visiteurs des programmes 
imprimés en braille, ce qui a permis à plusieurs déficients visuels 
de profiter pleinement de l'événement. Dans la continuité, en 2015 
nous souhaitons proposer la traduction de certaines conférences en LSF.
\end{minipage}

\newpage

\subsection{Des intervenants de qualité}

Le Capitole du Libre a accueilli des personnalités du Logiciel Libre telles que Adrienne \bsc{Charmet} ou Jérémie \bsc{Zimmermann} de \textbf{La Quadrature du Net}, Benjamin \bsc{Bayart} de \textbf{FDN}, Stéphane \bsc{Bortzmeyer} de l'\textbf{AFNIC}, Alix \bsc{Cazenave} et Frédéric \bsc{Couchet} de l'\textbf{April}, Claire \bsc{Gallon} de \textbf{LiberTIC}, Alexis \bsc{Kauffmann} et Pierre-Yves \bsc{Gosset} de \textbf{Framasoft}, Sandrine \bsc{Mathon} de \textbf{Toulouse Métropole}, Nicolas \bsc{Barcet} de \textbf{eNovance \& RedHat}, Lucas \bsc{Nussbaum} et Stefano \bsc{Zacchiroli} de \textbf{Debian}, François \bsc{Pelligrini}, Paul \bsc{Rouget} de la \textbf{Mozilla Fondation}, Christophe \bsc{Sauthier} d'\textbf{Ubuntu-fr}, etc.

\subsection{Des ateliers et animations pour expérimenter}

%\InsertImage{CDL2014-2953-imprimante3D-photo-jeanZ.JPG}{0.33334}{r}

\begin{minipage}{0.6\textwidth}
Et comme rien ne vaut l'expérimentation, tout au long du weekend sont 
proposées au public des animations et ateliers ludiques. Cela est 
l'occasion de découvrir les imprimantes 3D, comment héberger soi-même son 
site web, contribuer à OpenStreetMap, ou encore programmer son jeu vidéo.
\end{minipage}
\begin{minipage}{0.4\textwidth}
\photo{CDL2014-2953-imprimante3D-photo-jeanZ.JPG}{0.8\textwidth}
\end{minipage}



\section{Devenir \textit{Sponsor}}

	% Partie devenir sponsor

Le Capitole du Libre recherche des partenaires pour financer l'événement. Nous souhaitons conserver une manifestation libre et accessible au plus grand nombre. Nos financements nous servent notamment à couvrir les frais de déplacement de nos intervenants. Nous proposons à tout type d'entreprises de nous aider avec différents niveaux de sponsorings détaillés ci-dessous.

\Separateur

Nous sponsoriser, c'est vous associer à cette manifestation et soutenir le Logiciel Libre. Le Capitole du Libre est un lieu idéal pour venir découvrir de nouveaux horizons, récolter de nouvelles idées, découvrir de nouveaux talents.

	\subsection{Niveaux de sponsoring}

    \begin{center}
    \begin{tabular}{|r|c|c|c|c|c|}
        \hline  & Bronze & Argent & Or & Platine & Diamant \\
        \hline Contribution & \SI{250}{€} & \SI{600}{€} & \SI{1000}{€} & \SI{2000}{€} & \SI{4000}{€} \\
        \hline Limite & - & - & - & 4 & 2 \\
        \hline Logo sur l'affiche et le site web & \ding{'064} & \ding{'064} & \ding{'064} & \ding{'064} & \ding{'064}  \\
        \hline Badges sponsors & \ding{'064} & \ding{'064} & \ding{'064} & \ding{'064} & \ding{'064} \\
        \hline Dépôt d'offre d'emploi et de stage & \ding{'064} & \ding{'064} & \ding{'064} & \ding{'064} & \ding{'064} \\
        \hline Logo diffusé entre les conférences & & \ding{'064} & \ding{'064} & \ding{'064} & \ding{'064} \\
        \hline Logo au début de chaque vidéo & & & \ding{'064} & \ding{'064} & \ding{'064} \\
        \hline Logo sur les \textit{flyers} de l'événement & & & \ding{'064} & \ding{'064} & \ding{'064} \\
        \hline Texte sur le programme distribué & & & \nicefrac{1}{4} page & \nicefrac{1}{2} page & 1 page \\
        \hline Espace de présentation / Stand & & & Kakémono & 1 table & 2 tables \\
        \hline Remerciements lors de la conférence de clôture & & & & \ding{'064} & \ding{'064}  \\
        \hline Une idée ? Contactez-nous ! & & & & & \ding{'064} \\
        \hline Mise à disposition d'une salle privative & & & & & \ding{'064} \\
        \hline 
    \end{tabular}
    \end{center}

Note: sur les différents supports, la taille de votre logo sera proportionnelle à votre niveau de support.

	\subsection{Budget de l’événement}

Le budget total de l'édition 2014 du Capitole du Libre s'élevait à \SI{12000}{€}. Les différents postes de dépense sont indiqués dans le graphique ci-dessous:

\includegraphics[scale=0.6]{Images/budget_2014.png}\\

Pour l'édition 2015 du Capitole du Libre, nous prévoyons une augmentation de notre budget à \SI{18000}{€} afin principalement :
\begin{itemize}[label=$\bullet$]
\item d'augmenter la superficie de l'événement ;
\item de permettre de faire venir plus d'orateurs afin de continuer à améliorer la qualité et la variété de nos orateurs ;
\item de proposer un buffet dinatoire le samedi soir. Ce buffet ouvert à tous et à participation libre étant un lieu d'échange privilégié lors de ce week-end où se retrouvent orateurs, sponsors et participants.
\end{itemize}

\Separateur

Le détail des dépenses et des recettes du Capitole du Libre 2015 est détaillé dans les deux tableaux respectifs \ref{tab_dépenses} et \ref{tab_recettes} ci-dessous.

\begin{table}[!h]
\begin{center}
	\caption{Dépenses du Capitole du Libre 2015}\label{tab_dépenses}
    \begin{tabular}{|l|r|}
        \hline Dépense & Montant \\
        \hline \textbf{Défraiements intervenants} & \textbf{\SI{5000}{€}} \\
        \hline Déplacements intervenants & \SI{3500}{€} \\
        \hline Hébergement intervenants & \SI{1500}{€} \\
        \hline \textbf{Hébergement manifestation} & \textbf{\SI{3000}{€}}\\
        \hline Chauffage & \SI{1500}{€} \\
        \hline Sécurité & \SI{1500}{€} \\
        \hline \textbf{Apéritif \& Repas} & \textbf{\SI{6000}{€}}\\
        \hline Participation repas intervenants et bénévoles & \SI{1500}{€} \\
        \hline Buffet samedi soir & \SI{4500}{€} \\
        \hline \textbf{Buvette} & \textbf{\SI{600}{€}}\\
        \hline Viénoiseries bénévoles & \SI{150}{€} \\
        \hline Approvisionnement buvette & \SI{400}{€} \\
        \hline Location machines à café & \SI{50}{€} \\
        \hline \textbf{Goodies} & \textbf{\SI{2800}{€} }\\
        \hline T-Shirt Capitole du Libre 2015 & \SI{1400}{€} \\
        \hline Goodies autres & \SI{1400}{€} \\
        \hline \textbf{Communication} & \textbf{\SI{800}{€}} \\
        \hline Impression Flyers & \SI{100}{€} \\
        \hline Impression Affiches & \SI{100}{€} \\
        \hline Impression Programmes & \SI{500}{€} \\
        \hline Fléchage / Indications & \SI{100}{€} \\
        \hline
        \hline \textbf{TOTAL} & \textbf{\SI{18200}{€}} \\
        \hline
    \end{tabular}
\end{center}
\end{table}

\begin{table}[!h]
\begin{center}
    \caption{Recettes du Capitole du Libre 2015}\label{tab_recettes}
    \begin{tabular}{|c|r|}
        \hline Dépense & Montant \\
        \hline Sponsors & \SI{15900}{€} \\
        \hline Dons & \SI{300}{€} \\
        \hline Ventes buvette & \SI{600}{€} \\
        \hline Recette boutique & \SI{1600}{€} \\
        \hline \textbf{TOTAL} & \textbf{\SI{18200}{€}} \\
        \hline
    \end{tabular}
\end{center}
\end{table}



\section{La ville rose}

	% Partie la ville rose

Capitole du Libre se déroule tous les ans à Toulouse dans l'école d'ingénieur de l'ENSEEIHT (École Nationale Supérieure d'Élelectrotechnique, d'Electronique, d'Informatique, d'Hydraulique et des Télécommunications de Toulouse). 

\subsection{Se rendre à l'ENSEEIHT}

Toulouse est doté d'un réseau de bus, de deux lignes de métro ainsi que de
deux lignes de tramway. Des vélos sont disponibles en libre service auprès
de bornes spécifiques.

L'ENSEEIHT, se situe dans la rue riquet
 proche de la station de métro François Verdier. Elle est desservie
par les bus 27 (arrêt Guilhemery), 16 et 22 (arrêt Place Dupuy) ainsi que 2,
10, 14, 29 et 38 (arrêt François Verdier).

\subsubsection*{Depuis la gare}

Vous pouvez marcher le long du canal environ 20 mn ou encore prendre la
ligne de bus 27 en direction de Rangueil et descendre à l'arrêt Guilhemery.

\subsubsection*{Depuis l'aéroport}

Prenez la ligne de tramway T2 jusqu'au terminus (Arènes) puis prenez le métro
(ligne A) en direction de Balma-Gramont. Descendez à Jean-Jaurès et prenez
la correspondance sur la ligne de métro B en direction de Ramonville. Sortez
à la station François Verdier.

\subsection{Accès}

\InsertPhoto{Images/carte_acces.png}{Contributeurs de OpenStreetMap. Tiles courtesy of Andy Allan -- CC-BY-SA 2.0}


	
\section{Contact}

	% Partie contact


Pour toutes questions relatives au sponsoring du Capitole du Libre :
\begin{itemize}
\item[\logo] écrivez à \href{mailto:contact@capitoledulibre.org}{\nolinkurl{contact@capitoledulibre.org}}
\end{itemize}
\paragraph{Contact presse :}
\begin{itemize}
\item[\logo] écrivez à \href{mailto:comm@capitoledulibre.org}{\nolinkurl{comm@capitoledulibre.org}}
\end{itemize}


\section{Sources \& références}

	% Partie sources & références

\vfill
\begin{center}
\textcolor{Cdl}{Crédits images} \par
{\tiny
\textbf{Photo Capitole} : \href{https://www.flickr.com/photos/blieusong/6986608500/in/set-72157629942158013}{Benh \bsc{Lieu Song} -- CC-BY-SA~3.0} via Flickr $\bullet$ \textbf{Photo conférence \bsc{Bortzmayer}} : Guillaume Paumier CC--BY~2.0 $\bullet$ \textbf{Photo hall N7} : ?? $\bullet$ \textbf{Photo clavier braille} : ?? $\bullet$ \textbf{Photo imprimante 3D} : ?? $\bullet$ \textbf{Logo \bsc{Gnu}} : \href{https://commons.wikimedia.org/wiki/File\%3AOfficial_gnu.svg"><img width="512" alt="Official gnu" src="//upload.wikimedia.org/wikipedia/commons/thumb/3/39/Official_gnu.svg/512px-Official_gnu.svg.png}{“\bsc{Gnu} Project logo”} de \href{mailto:vcopovi@wanadoo.fr}{Victor Siame} via Wikimedia Commons $\bullet$ \textbf{Logo Audacity} : \href{https://commons.wikimedia.org/wiki/File:Audacity_Logo.svg#/media/File:Audacity_Logo.svg}{“Audacity logo”} by Aaron Spike -- Part of Audacity source code released under GPLv2.. Licensed under GPL via Wikimedia Commons $\bullet$ \textbf{Logo Thunderbird} : ?? $\bullet$ Logo The~Gimp : \href{https://commons.wikimedia.org/wiki/File:The_GIMP_icon_-_gnome.svg#/media/File:The_GIMP_icon_-_gnome.svg}{“The GIMP icon -- Gnome”} by The GIMP's art/developer team -- The GIMP package. Licensed under GPL via Wikimedia Commons $\bullet$ \textbf{Logo LibreOffice} : ?? $\bullet$ \textbf{Logo VLC} : \href{https://commons.wikimedia.org/wiki/File:VLC_Icon.svg#/media/File:VLC_Icon.svg}{“VLC Icon”} by Richard C. G. \bsc{Øiestad} -- \url{http://www.videolan.org} $\bullet$ \textbf{Logo Firefox} : ??
%Licensed under GPL via Wikimedia Commons


}
\end{center}



\end{document}


%%%%%%%%% OLD DOSSIER SPONSORING CDL 2014 %%%%%%%%%%%%%%%%%%%%%%%%%%%%%%%%%

%\CreerTitre{Images/titre2.jpg}{Benh \bsc{Lieu Song} -- CC-BY-SA 3.0}

%\begin{Introduction}

%L'association \textbf{Toulibre} organise le \textcolor{Cdl}{15 et 16 novembre 2014} la quatrième édition du \textbf{\g{Capitole du Libre}}, un événement consacré aux Logiciels Libres, orienté à la fois vers le grand public et le public spécialisé.
%
%\Separateur
%
%Des cycles de conférences grand public, techniques et multimédia ont lieu le samedi. Le dimanche est consacré à des ateliers pratiques.
%
%\Separateur
%
%Une \textbf{\textit{Install Party}} permet à tous de découvrir et d'installer un système Libre sur son ordinateur. Des stands de démo et d'animations sont proposés au public toute la journée du samedi. Un \textbf{\g{village du libre}} permet aux associations autour du libre de présenter leur activité.
%
%\Separateur
%
%Le \textbf{Capitole du Libre} est également l'occasion de réunir des communautés du Libre pour des conférences, \textit{lightning talks}, \textit{coding sprints}\dots ~ Le \textbf{Capitole du Libre} a accueilli plusieurs conférences depuis 2011 telles que \textbf{DrupalCamp}, \textbf{DjangoCon},  \textbf{FranceJS}, \textbf{LuaWorkshop}, \textbf{OpenStack} et \textbf{Akademy-FR}.
%
%\Separateur
%
%Vous trouverez plus d'informations sur le site internet : \url{\Web}.
%\end{Introduction}
%
%\section{Conférences, démonstrations et ateliers du Capitole du Libre}
%
%Pendant tout un weekend, le Capitole du Libre propose conférences, animations et ateliers autour du Logiciel Libre et du bien commun, et accueille des événements de la communauté du Libre.
%
%\subsection{Conférences et ateliers}
%
%Les conférences et les ateliers qui seront programmés pourront aborder les thèmes suivants :
%\begin{itemize}
%\item[\logo] un cycle de \textbf{conférences grand public} couvrant des sujets comme les enjeux des Logiciels Libres, le Libre au-delà du Logiciel, Wikipédia, les aspects économiques ou sociaux du Logiciel Libre\dots ~ ;
%\item[\logo] un thème \textbf{bureautique et multimédia}, couvrant des sujets tels que la retouche d'image, la modélisation 3D, la musique assistée par ordinateur, la bureautique\dots ~ ;
%\item[\logo] un thème \textbf{technique} couvrant des sujets de développement logiciels, d'embarqué, d'administration système ou réseau\dots ~ ;
%\item[\logo] un thème \textbf{Internet Libre} couvrant les solutions qui permettent de maîtriser ses données personnelles sur la Toile ;
%\item[\logo] un thème \textbf{Arduino} et \textbf{Open Hardware} sur les aspects matériel et montages électroniques libres ;
%\item[\logo] un thème \textbf{DevOps} sur les problématiques d'automatisation du déploiement d'applications.
%\end{itemize}
%
%\Separateur
%
%Dans chaque thème, les conférences proposées seront de 20 minutes à une heure, et permettent de parler de plus de sujets dans une durée adaptée.
%
%\Separateur
%
%Depuis l'édition de 2011, les conférences ont lieu le samedi, et le dimanche est consacré aux ateliers pratiques, aussi bien pour le grand public que pour un public averti. Les thèmes pourront évoluer en fonction des propositions reçues.
%
%\Separateur
%
%Les années précédentes sont notamment intervenus Nicolas \bsc{Barcet} de \textbf{eNovance \& RedHat}, Benjamin \bsc{Bayart} de \textbf{FDN}, Stéphane \bsc{Bortzmeyer} de l'\textbf{AFNIC}, Adrienne \bsc{Charmet Alix} de \textbf{Wikimedia France}, Alix \bsc{Cazenave} et Frédéric \bsc{Couchet} de l'\textbf{April}, Claire \bsc{Gallon} de \textbf{LiberTIC}, Alexis \bsc{Kauffmann} et Pierre-Yves \bsc{Gosset} de \textbf{Framasoft}, Sandrine \bsc{Mathon} de \textbf{Toulouse Métropole}, Lucas \bsc{Nussbaum} et Stefano \bsc{Zacchiroli} de \textbf{Debian}, François \bsc{Pelligrini}, Paul \bsc{Rouget} de la \textbf{Mozilla Fondation}, Christophe \bsc{Sauthier} d'\textbf{Ubuntu-fr}, Jérémie \bsc{Zimmermann} de \textbf{La Quadrature du Net}\dots
%
%\subsection{Espace de stands et d'échanges}
%
%Lieu de passage du public, le grand hall de l'\bsc{Enseeiht} est un espace dédié aux stands pour les organisations, associations ou entreprises.
%
%\Separateur
%
%Il est possible également de poser du matériel de communication tel que \g{kakemono}, présentoirs\dots
%
%\Separateur
%
%En plus des stands, un espace convivial sera aménagé afin de permettre rencontres et discussions.
%
%\newpage
%\subsection{Install Party, Lan Party et espace démonstrations}
%
%L'\textit{install party} est un événement de promotion et de démocratisation des Logiciels Libres auprès du grand public, et un événement de rencontre des acteurs de la communauté du Logiciel Libre.
%
%\Separateur
%
%Organisé chaque année depuis 2008, il est intégré dans le Capitole du Libre depuis 2011.
%
%\Separateur
% 
%Cet événement propose :
%\begin{itemize}
%\item[\logo] un espace de démonstration de Logiciels Libres, où les visiteurs pourront poser leurs questions ou participer à des mini-ateliers ;
%\item[\logo] Des mini-présentations des différentes distributions Libres proposées (Ubuntu, Fedora, OpenSuse, LinuxMint\dots) ;
%\item[\logo] une \textit{install party}, permettant au grand public de trouver de l'aide pour installer des logiciels et distributions Libres sur leur propre ordinateur.
%\end{itemize}
%
%\subsection{Nouveautés de cette édition}
%
%Grande nouveauté du Capitole du Libre 2014 : un \textcolor{Cdl}{village associatif} va être organisé permettant ainsi de présenter de nombreuses distributions \bsc{Gnu}/Linux. Plusieurs groupes d'utilisateurs du libre seront aussi au rendez-vous pour présenter au public les dernières nouveautés technologiques libres du moment.
%
%\section{Événements hébergés}
%
%\subsection{Akademy-fr}
%
%\ImageDroitebis{Images/klogo-official-lineart_simple-128x128.png}
%%
%Akademy-fr est la déclinaison française de la conférence KDE annuelle. KDE est une communauté internationale produisant un ensemble d'applications multi-plateformes et notamment un environnement de bureau nommé Plasma Desktop.
%
%\Separateur
%
%Le but de l'Akademy-fr est d'améliorer la promotion de KDE au niveau de la France, à l'aide de conférences en français et d'ateliers orientés contribution.
%
%\subsection{Hackfest LibreOffice}
%
%\ImageDroite{Images/LO.png}
%
%LibreOffice est une suite bureautique libre et gratuite ; son interface claire et ses puissants outils vous permettent de libérer votre créativité et de développer votre productivité.
%
%\Separateur
%
%LibreOffice intègre plusieurs applications qui en font la plus puissante suite bureautique Libre et Open Source du marché. Le développement est ouvert à de nouveaux talents et de nouvelles idées, et le logiciel est testé et utilisé quotidiennement par une importante communauté d'utilisateurs dévoués.
%
%\Separateur
%
%Dans un \textit{Hackfest} les contributeurs LibreOffice se réunissent afin de coordonner dans un temps donné et dans une atmosphère détendue, le développement du produit et faire avancer le projet. Au sein de LibreOffice, cela passe par une communauté internationale de développeurs, designers, traducteurs et \g{QA triagers}.
%
%\Separateur 
%
%C'est une occasion idéale pour commencer ou continuer à participer à l'un des plus grands projets Open Source dans l'action !
%
%\section{Conditions de sponsoring}
%
%\textbf{Toulibre} offre la possibilité à des entreprises d'associer leur nom à l'événement \g{Capitole du Libre} qui touchera à la fois le grand public et le public spécialisé en informatique. Auprès du grand public, l'image des sponsors sera associée à un événement s'intéressant aux enjeux éthiques et sociaux du numérique. Auprès du public spécialisé, les sponsors se feront connaître comme acteurs du monde du Logiciel Libre.\par 
%Vous pouvez participer à l'événement financièrement ou bien en prenant en charge un poste de dépense (comme l'impression des affiches, le repas du dimanche\dots).
%
%\Separateur
%
%Selon que l'entreprise qui sponsorise est un grand compte ou une PME, le niveau des montants est différent (que se soit en service ou en monétaire).
%
%\Separateur
%
%En échange du sponsoring du Capitole du Libre par les entreprises associées à l'événement, Toulibre s'engage à :
%
%\begin{center}
%\arrayrulecolor{Cdl} 
%\begin{tabular}{|p{10cm}ccc|}
%\hline
%\rowcolor{Cdl} & \textbf{Bronze} & \textbf{Argent} & \textbf{Or} \\
%\hline\hline
%{\hfill\textit{Grand compte}} & \SI{300}{\euro} & \SI{700}{\euro} & \SI{1500}{\euro} \\ 
%
%{\hfill\textit{PME}} & \SI{200}{\euro} & \SI{500}{\euro} & \SI{900}{\euro} \\ 
%\hline\hline
%Logo sur l'affiche de l'événement\textcolor{Cdl}{*} & \textcolor{Cdl}{\ding{'064}} & \textcolor{Cdl}{\ding{'064}} & \textcolor{Cdl}{\ding{'064}} \\ 
%
%Logo sur le site internet dédié à l'événement\textcolor{Cdl}{*} & \textcolor{Cdl}{\ding{'064}} & \textcolor{Cdl}{\ding{'064}} & \textcolor{Cdl}{\ding{'064}} \\ 
%
%Logo au début de chaque vidéo\textcolor{Cdl}{*} & \textcolor{Cdl}{\ding{'064}} & \textcolor{Cdl}{\ding{'064}} & \textcolor{Cdl}{\ding{'064}} \\ 
%
%Badges sponsors & \textcolor{Cdl}{\ding{'064}} & \textcolor{Cdl}{\ding{'064}} & \textcolor{Cdl}{\ding{'064}} \\ 
%
%Remerciements lors de la conférence de clôture de l'événement & \textcolor{Cdl}{\ding{'064}} & \textcolor{Cdl}{\ding{'064}} & \textcolor{Cdl}{\ding{'064}} \\ 
%
%Logo sur le programme papier distribué aux participants\textcolor{Cdl}{*} &  & \textcolor{Cdl}{\ding{'064}} & \textcolor{Cdl}{\ding{'064}} \\ 
%
%Logo sur les flyers de l'événement\textcolor{Cdl}{*} &  & \textcolor{Cdl}{\ding{'064}} & \textcolor{Cdl}{\ding{'064}} \\ 
%
%
%Texte de description sur le programme papier distribué\textcolor{Cdl}{**} &  & \textcolor{Cdl}{\textbf{120}} & \textcolor{Cdl}{\textbf{380}} \\
%
%Espace pour un stand &  &  & \textcolor{Cdl}{\ding{'064}} \\ 
%
%Interview sur le site du Capitole du Libre &  &  & \textcolor{Cdl}{\ding{'064}} \\
%\hline
%\multicolumn{4}{r}{\textcolor{Cdl}{*} \textit{logo de taille différente : or plus grand que argent plus grand que bronze}}\\
%\multicolumn{4}{r}{\textcolor{Cdl}{**} \textit{en nombre de caractères}}\\
%
%\end{tabular}
%\end{center}
%
%\subsection{Les dépenses}
%
%Comme les années passées, les conférenciers pour les présentations et les ateliers seront défrayés. Votre participation permet également de financer le chauffage des locaux de l'\bsc{Enseeiht} que nous occupons tout le weekend, ainsi qu'un poste de secours obligatoire pour un événement de cette ampleur. L'apéritif dînatoire du samedi soir est en partie financé par la participation libre des convives. Tout le public y est invité.
%
%\Separateur
%
%Le Capitole du Libre est organisé exclusivement par les \textbf{bénévoles} de l'association \textbf{Toulibre} et des \textbf{clubs info, vidéo et animation de l'\bsc{ENSEEIHT}}.
%
%\Separateur
%Le budget du Capitole du Libre s'élève à environ \SI{10000}{\euro}.
%
%\newpage
%\subsection{Ils nous ont soutenu en 2013}
%
%\begin{center}
%{\Large \textcolor{Cdl}{Partenaires Or}}
%
%\urllogoau{http://www.enovance.com}{enovance.png}\hspace{1cm}
%\urllogoau{www.kdab.com}{kdab.png}\hspace{1cm}
%\urllogoau{http://www.logilab.fr}{logilab.png} \\
%\urllogoau{http://makina-corpus.com}{makina-corpus.png}\hspace{1cm}
%\urllogoau{http://boards.openpandora.org/page/homepage.html}{openpandora.png}\hspace{1cm}
%\urllogoau{http://www.sierrawireless.com}{sierra-wireless.png}
%
%\Separateur
%
%{\Large \textcolor{Cdl}{Partenaires Argent}}
%
%\Separateur
%
%\urllogoag{http://www.objectif-libre.com}{objectif-libre.png}
%
%\Separateur
%
%{\Large \textcolor{Cdl}{Partenaires Bronze}}
%
%\Separateur
%
%\urllogobr{http://blue-mind.net}{bluemind.jpg} \hspace{1cm}
%\urllogobr{http://free-electrons.com}{free-electrons.png}\hspace{1cm}
%\urllogobr{http://www.nfrance.com}{nfrance-conseil.png} \\
%\urllogobr{http://osones.com}{osones.png} \hspace{1cm}
%\urllogobr{http://www.solulibre.com}{solulibre.png}
%\end{center}
%
%
%\section{Contact}
%
%Pour toutes questions relatives au sponsoring du Capitole du Libre :
%\begin{itemize}
%\item[\logo] écrivez à \href{mailto:contact@capitoledulibre.org}{\nolinkurl{contact@capitoledulibre.org}}
%\end{itemize}
%\paragraph{Contact presse :}
%\begin{itemize}
%\item[\logo] écrivez à \href{mailto:comm@capitoledulibre.org}{\nolinkurl{comm@capitoledulibre.org}}
%\end{itemize}
%
%\paragraph{Photos première page  :} \href{https://commons.wikimedia.org/wiki/File\%3AToulouse_Capitole_Night_Wikimedia_Commons.jpg}{Benh \bsc{Lieu Song} -- CC-BY-SA 3.0}



