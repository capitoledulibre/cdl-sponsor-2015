% Partie logiciels libres
% macros personnelles 
\newcommand{\lls}{logiciels libres~}
%%%%%%%%%%%%%%%%%%%%%%%%%%%%%%%%%%%%%%%%%%%%%%%%%%%%%%%%%%%

\citation{Je peux expliquer le logiciel libre en trois mots : \\ Liberté, Égalité, Fraternité.}{Richard Matthew \bsc{Stallman}}

\subsection{Quelques points d’histoire}
\begin{minipage}{0.7\textwidth}
C’est aux États-Unis en 1984 que la notion de \textcolor{Cdl}{\textit{logiciel libre}} a été définir par \textcolor{Cdl}{Richard Matthew \bsc{Stallman}} pour la première fois. Il rédigea en parallèle la \textcolor{Cdl}{\textit{licence publique générale \bsc{Gnu}}} (la \g{GPL}) permettant d’assurer et de protéger les libertés définies par le logiciel libre. Enfin, il fonde la même année la \textcolor{Cdl}{\textit{Free Software Fondation}} afin de baser son projet sur de solides bases.
\end{minipage}
\begin{minipage}{0.3\textwidth}
\begin{center}
\image{gnu.pdf}{0.8\textwidth}\\
\textit{Logo du projet \bsc{Gnu}}
\end{center}
\end{minipage}

\begin{multicols}{2}[\subsection{La philosophie des logiciels libres}]
La philosophie des \lls est le respect des libertés des utilisateurs. En les utilisant, ils ont la liberté de les \textbf{exécuter}, les \textbf{copier}, les \textbf{distribuer}, les \textbf{modifier} et enfin de les \textbf{améliorer}.

Il ne faut pas toutefois confondre logiciel gratuit et logiciel libre. Dans le premier cas, celui-ci n’assure pas forcement les quatre libertés de l’utilisateur. Les exemples les plus parlant sont certains logiciels de Google.

\textit{A contrario}, un  logiciel libre n’est pas obligatoirement gratuit. Il existe un modèle économique viable basé sur le développement ou la vente de service sur du logiciel libre. Un bon exemple est l’entreprise Red\,Hat aux États-Unis ou encore Bluemind en France.
\end{multicols}

\citationen{Think of ‘free speech’, not ‘free beer’.}{Richard Matthew \bsc{Stallman}}

Pour qu’un logiciel soit libre, celui-ci doit respecter quatre libertés essentielles :

\begin{center}
\begin{minipage}{0.8\textwidth}
\begin{center}
\textcolor{Cdl}{\large Liberté 0} \\ \textit{La liberté d'exécuter le programme comme vous voulez, pour n'importe quel usage.} \\
\textcolor{Cdl}{\large Liberté 1} \\ \textit{La liberté d'étudier le fonctionnement du programme, et de le modifier pour qu'il effectue vos tâches informatiques comme vous le souhaitez.} \\
\textcolor{Cdl}{\large Liberté 2} \\ \textit{La liberté de redistribuer des copies, donc d'aider votre voisin.} \\
\textcolor{Cdl}{\large Liberté 3} \\ \textit{La liberté de distribuer aux autres des copies de vos versions modifiées.}
\end{center}
\end{minipage}
\end{center}

\begin{multicols}{2}
La \textcolor{Cdl}{liberté 0} est essentielle pour une utilisation universelle du logiciel. Le logiciel doit pouvoir être disponible sur toutes les plateformes, et utilisable par tous.

Pour respecter les \textcolor{Cdl}{libertés 1 à 3}, le code source, ou la \g{recette}, du logiciel doit être mis à disposition des utilisateurs. Celui-ci permet à tous de pouvoir lire, étudier, modifier et améliorer le logiciel. 

L’accès aux \textcolor{Cdl}{sources} du logiciel permet aussi un contrôle des utilisateurs sur le travail du ou des créateur. Grâce à ce principe de vérification, les erreurs ou failles de sécurité peuvent être très rapidement corrigées. Pour cette raison, les logiciels libres font références en terme de sécurité et de fiabilité.
\end{multicols}

\begin{multicols}{2}[\subsection{Les formats ouverts}]
Le format des fichier est aussi une composante essentielle au développement des logiciels libres.

Un \textcolor{Cdl}{format ouvert} assure dans un premier temps la pérennité du fichier sur le long terme, et ce indépendamment d’un éditeur de logiciel. De ce fait, il sera toujours possible de lire un format ouvert, même si le logiciel initialement prévu pour le lire n’existe plus. Les formats ouverts permettent aussi la polyvalence des logiciels compatibles. L’utilisateur n’est pas obligé d’utiliser un logiciel spécifique ou encore d’utiliser un système d’exploitation particulier.
\end{multicols}

\subsection{Quelques exemples de logiciels libres}

\begin{minipage}{0.1\textwidth}
\begin{center}
\image{firefox.pdf}{0.6\textwidth} \\\vspace{1mm}
\image{libreoffice.pdf}{0.6\textwidth}\vspace{1mm}
\image{thunderbird.pdf}{0.6\textwidth}\vspace{1mm}
\image{vlc.pdf}{0.6\textwidth}\vspace{1mm}
\image{gimp.pdf}{0.6\textwidth}
\end{center}
\end{minipage}
\begin{minipage}{0.75\textwidth}
\begin{multicols}{2}
Les logiciels libres sont utilisés par énormément de monde, parfois sans qu’ils en aient connaissance. 

On peut par exemple citer le navigateur Web \textcolor{Cdl}{Firefox}, le logiciel de rédaction de documents \textcolor{Cdl}{LibreOffice}, le client de courriels \textcolor{Cdl}{Thunderbird}, le lecteur multimédia \textcolor{Cdl}{VLC} ou encore le logiciel de traitement d’images \textcolor{Cdl}{The Gimp}.

Les logiciels libres sont aussi utilisés en très grande majorité sur les serveurs des entreprises très souvent basées sur des distribution \bsc{Gnu}/Linux comme \textcolor{Cdl}{Debian} ou \textcolor{Cdl}{Red\,Hat}. Les serveurs utilisent par exemple des logiciels comme \textcolor{Cdl}{Apache}, \textcolor{Cdl}{MySQL} ou encore \textcolor{Cdl}{PHP}, représentant la quasi-totalité des sites Internet.
\end{multicols}
\end{minipage}
\begin{minipage}{0.15\textwidth}
\begin{center}
\image{debian.pdf}{0.6\textwidth}\vspace{1mm}
\image{redhat.pdf}{0.6\textwidth}\vspace{1mm}
\image{apache.pdf}{0.6\textwidth}\vspace{1mm}
\image{mysql.pdf}{0.6\textwidth}\vspace{1mm}
\image{php.pdf}{0.6\textwidth}
\end{center}
\end{minipage}
\vspace*{\baselineskip}
\begin{multicols}{2}[\subsection{Le libre au-delà du logiciel}]
La philosophie du logiciel libre restée inchangée depuis sa création s’est cependant beaucoup développée et s’est adaptée à d’autres médias ou activité. On dénombre désormais plusieurs autres domaines basée sur le même principe :
\begin{itemize}[label=$\bullet$]
\item la \textcolor{Cdl}{culture libre} : des écrivains, des musiciens ou encore des photographes partageant sous licence libre leur travail ;
\item le \textcolor{Cdl}{matériel libre} : suite au développement récent des imprimantes 3D, de nombreuses personnes distribuent sous licence libre des objets 3D. Certaines imprimantes sont elles-même développées par des communautés et mises à disposition sous licence libre.
\item des \textcolor{Cdl}{projets libres} : les plus connus sont l’encyclopédie libre Wikipédia ou encore le site de cartographie, \mbox{OpenStreetMap}.
\end{itemize}
\end{multicols}

