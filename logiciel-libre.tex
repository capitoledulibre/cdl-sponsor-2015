% Partie logiciels libres
% macros personnelles 
\newcommand{\lls}{logiciels libres~}
%%%%%%%%%%%%%%%%%%%%%%%%%%%%%%%%%%%%%%%%%%%%%%%%%%%%%%%%%%%

\citation{Je peux expliquer le logiciel libre en trois mots : \\ Liberté, Égalité, Fraternité.}{Richard Matthew \bsc{Stallman}}

\begin{minipage}{0.7\textwidth}
En 1984 \textcolor{Cdl}{Richard Matthew \bsc{Stallman}} rédige la \textcolor{Cdl}{\textit{licence publique générale \bsc{Gnu}}} (la \g{GPL}) permettant d’assurer et de protéger les libertés définies par le logiciel libre. 
\end{minipage}
\begin{minipage}{0.3\textwidth}
\begin{center}
\image{gnu.pdf}{0.8\textwidth}\\
\textit{Logo du projet \bsc{Gnu}}
\end{center}
\end{minipage}

\begin{multicols}{2}[\subsection{La philosophie des logiciels libres}]
La philosophie des \lls est le respect des libertés des utilisateurs. En les utilisant, ils ont la liberté de les \textbf{exécuter}, les \textbf{copier}, les \textbf{distribuer}, les \textbf{modifier} et enfin de les \textbf{améliorer}.

Il ne faut toutefois pas confondre logiciel gratuit et logiciel libre. Dans le premier cas, celui-ci n’assure pas forcement les quatre libertés de l’utilisateur. Les exemples les plus parlant sont certains logiciels de Google.

\textit{A contrario}, un  logiciel libre n’est pas obligatoirement gratuit. Il existe un modèle économique viable basé sur le développement ou la vente de service sur du logiciel libre. Un bon exemple est l’entreprise Red\,Hat aux États-Unis ou encore Bluemind en France.
\end{multicols}

\citationen{Think of ‘free speech’, not ‘free beer’.}{Richard Matthew \bsc{Stallman}}

Pour qu’un logiciel soit libre, celui-ci doit respecter quatre libertés essentielles :

\begin{center}
\begin{minipage}{0.8\textwidth}
\begin{center}
\textcolor{Cdl}{\large Liberté 0} \\ \textit{La liberté d'exécuter le programme comme vous voulez, pour n'importe quel usage.} \\
\textcolor{Cdl}{\large Liberté 1} \\ \textit{La liberté d'étudier le fonctionnement du programme, et de le modifier pour qu'il effectue vos tâches informatiques comme vous le souhaitez.} \\
\textcolor{Cdl}{\large Liberté 2} \\ \textit{La liberté de redistribuer des copies, donc d'aider votre voisin.} \\
\textcolor{Cdl}{\large Liberté 3} \\ \textit{La liberté de distribuer aux autres des copies de vos versions modifiées.}
\end{center}
\end{minipage}
\end{center}

La \textcolor{Cdl}{liberté 0} est essentielle pour une utilisation universelle du logiciel. Le logiciel doit pouvoir être disponible sur toutes les plateformes, et utilisable par tous.

\Separateur

Pour respecter les \textcolor{Cdl}{libertés 1 et 3}, le code source, ou la \g{recette}, du logiciel doit être mis à disposition des utilisateurs. Celui-ci permet à tous de pouvoir lire, étudier, modifier et améliorer le logiciel.

\Separateur

L’accès aux \textcolor{Cdl}{sources} du logiciel permet aussi un contrôle des utilisateurs sur le travail du ou des créateurs. Grâce à ce principe de vérification, les erreurs ou failles de sécurité peuvent être très rapidement corrigées. Pour cette raison, les logiciels libres font références en terme de sécurité et de fiabilité.

\subsection{Les formats ouverts}

Le format des fichier est aussi une composante essentielle au développement des logiciels libres.

Un \textcolor{Cdl}{format ouvert} assure la pérennité du fichier sur le long terme, et ce indépendamment d’un éditeur de logiciel. De ce fait, il sera toujours possible de lire un format ouvert, même si le logiciel initialement prévu pour le lire n’existe plus. Les formats ouverts permettent aussi la polyvalence des logiciels compatibles. L’utilisateur n’est pas obligé d’utiliser un logiciel spécifique ou encore d’utiliser un système d’exploitation particulier.

\subsection{Quelques exemples de logiciels libres}

\begin{minipage}{0.7\textwidth}
Dans la vie courante on peut citer le navigateur Web \textcolor{Cdl}{Firefox}, 
le logiciel de rédaction de documents \textcolor{Cdl}{LibreOffice}, le 
client de courriels \textcolor{Cdl}{Thunderbird}, le lecteur 
multimédia \textcolor{Cdl}{VLC} ou encore le logiciel de traitement 
d’images \textcolor{Cdl}{The Gimp}.

\Separateur

Les logiciels libres sont également utilisés en grande majorité sur 
les serveurs des entreprises, qui utilisent par exemple des 
logiciels comme \textcolor{Cdl}{Apache}, \textcolor{Cdl}{MySQL} ou 
encore \textcolor{Cdl}{PHP} pour les plus connus.
\end{minipage}
\begin{minipage}{0.3\textwidth}
\begin{center}
\includegraphics[scale=0.9]{logos-logiciels-libres.png}
\end{center}
\end{minipage}

\subsection{Le libre au-delà du logiciel}

La philosophie du logiciel libre s’est étendue à d’autres domaines,
 notamment :

\begin{minipage}{0.3\textwidth}
\begin{center}
\image{fin-cc-by-scarlett-san-martin.jpg}{1\textwidth}
\textit{Meuble designé par Scarlett San Martin, CC--BY}
\end{center}
\end{minipage}
\begin{minipage}{0.7\textwidth}
\begin{itemize}[label=$\bullet$]
\item l'\textcolor{Cdl}{art} et la \textcolor{Cdl}{culture libre} : 
des écrivains, des musiciens 
ou encore des photographes partageant sous licence libre leur travail ;
\item de la \textcolor{Cdl}{connaissance libre} : le projet le plus connu est 
l’encyclopédie libre Wikipédia, et aujourd'hui des universités 
publient leurs supports de cours sous licence libre. 
\item l'\textcolor{Cdl}{Open data} : Toulouse est pionnière en la matière, 
et est aujourd'hui présidente de Open Data France. On peut également 
citer le site de cartographie \mbox{OpenStreetMap} ;
\item le \textcolor{Cdl}{matériel libre} : suite au développement 
récent des imprimantes 3D, de plus en plus de plans sont disponibles
 sous licence libre : des meubles, des éoliennes, voire des véhicules.
\end{itemize}
\end{minipage}


